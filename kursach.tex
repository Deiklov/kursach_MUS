\documentclass[a4paper,14pt]{article}


%%% Работа с русским языком
\usepackage{cmap}					% поиск в PDF
\usepackage[T2A]{fontenc}			% кодировка
\usepackage[utf8]{inputenc}		% кодировка исходного текста
\usepackage[russian]{babel}	% локализация и переносы
%\usepackage{pscyr}
%\renewcommand{\rmdefault}{ftm}
%%% Дополнительная работа с математикой
\usepackage{amsmath,amsfonts,amssymb,amsthm,mathtools} % AMS

%% Номера формул
%\mathtoolsset{showonlyrefs=true} % Показывать номера только у тех формул, на которые есть \eqref{} в тексте.
%\usepackage{leqno} % Нумерация формул слева
%\usepackage{rumathgrk1}
%% Перенос знаков в формулах (по Львовскому)
\newcommand*{\hm}[1]{#1\nobreak\discretionary{}
	{\hbox{$\mathsurround=0pt #1$}}{}}
%\usepackage{glonti}
%%% Работа с картинками
\usepackage{graphicx}  % Для вставки рисунков
\graphicspath{{images/}{images2/}}  % папки с картинками
\usepackage{wrapfig} % Обтекание рисунков текстом
\addto\captionsrussian{\def\refname{Список используемой литературы}}
%%% Работа с таблицами
\usepackage{array,tabularx,tabulary} % Дополнительная работа с таблицами
\usepackage{longtable}  % Длинные таблицы
\usepackage{multirow} % Слияние строк в таблице
%%% Теоремы
\theoremstyle{plain} % Это стиль по умолчанию, его можно не переопределять.
\newtheorem{theorem}{Теорема}[section]
\newtheorem{proposition}[theorem]{Утверждение}

\theoremstyle{definition} % "Определение"
\newtheorem{corollary}{Следствие}[theorem]
\newtheorem{problem}{Задача}[section]

\theoremstyle{remark} % "Примечание"
\newtheorem*{nonum}{Решение}
%\pagestyle{empty}
%%% Страница
\usepackage{extsizes} % Возможность сделать 14-й шрифт
\usepackage{geometry} % Простой способ задавать поля
\geometry{top=20mm}
%\geometry{bottom=35mm}
\geometry{left=25mm}
\geometry{right=20mm}
\setlength{\parindent}{1.1cm}

\usepackage{setspace} % Интерлиньяж
\onehalfspacing % Интерлиньяж 1.5
% \doublespacing % Интерлиньяж 2
%\singlespacing % Интерлиньяж 1

\usepackage{lastpage} % Узнать, сколько всего страниц в документе.
\usepackage[usenames]{color}
\usepackage{colortbl}
\renewcommand{\baselinestretch}{1.05}
\usepackage{hyperref}
\usepackage[usenames,dvipsnames,svgnames,table]{xcolor}
\hypersetup{				% Гиперссылки
	unicode=true,           % русские буквы в раздела PDF
	pdftitle={Заголовок},   % Заголовок
	pdfauthor={Автор},      % Автор
	pdfsubject={Тема},      % Тема
	pdfcreator={Создатель}, % Создатель
	pdfproducer={Производитель}, % Производитель
	pdfkeywords={keyword1} {key2} {key3}, % Ключевые слова
	colorlinks=true,       	% false: ссылки в рамках; true: цветные ссылки
	linkcolor=black,          % внутренние ссылки
	citecolor=blue,        % на библиографию
	filecolor=magenta,      % на файлы
	urlcolor=cyan           % на URL
}

\usepackage{bm}

\begin{document}
% НАЧАЛО ТИТУЛЬНОГО ЛИСТА
\begin{center}
    {\textsc{Федеральное государственное бюджетное образовательное
            учреждение высшего образования
        }}\\
    {\textsc{Московский государственный университет имени М.В. Ломоносова
    }} \\
    \vspace{0.2cm}
    {\textsc{Механико - математический факультет}}\\
    \vspace{0.2cm}
    {\textsc{Кафедра прикладной механики и управления}}\\
    \hfill \break
    \begin{figure}[h!]
        \centering
        \includegraphics[width=0.30\linewidth]{emblema}
        \label{fig:emblema}
    \end{figure}
    \hfill \break
    \hfill \break
    \large{\textbf{Курсовая работа}\\
        \hfill \break Модель восстановления человеком исходной позы после толчка
    }
\end{center}

\hfill \break
\hfill \break
\begin{flushright}
    {
        Выполнил: студент группы М -- 1 \\ Романов Андрей Владимирович}
\end{flushright}

\begin{flushright}
    {
        Научный руководитель: к.ф.-м.н., \\ Кручинин Павел Анатольевич}
\end{flushright}
\hfill \break
\hfill \break
\begin{center} {Москва, 2022} \end{center}



\thispagestyle{empty} % выключаем отображение номера для этой страницы
\normalsize{
% КОНЕЦ ТИТУЛЬНОГО ЛИСТА
\newpage
\tableofcontents
\newpage

\section{Введение}
Проба со ступенчатым воздействием является одной из стандартных проб
при стабилометрических исследованиях. При проведении этой пробы
обследуемый стоит на платформе стабилоанализатора перед экраном, на
котором изображена мишень и отображается движение центра давления
человека, определяемое по показаниям стабилоанализатора.

В ходе теста производят толкающее воздействие на человека с помощью руки или
груза, помещенного на подвижном отвесе. В результате внешнего
воздействия тело человека наклоняется вперед и при не очень сильном толчке
он не теряет равновесие и не падает, а возвращается в исходное
положение за счет изменения угла в голеностопном суставе. Изменение
остальных суставных углов может оказаться тоже не столь значительным.
Родственные задачи уже решались в работах \cite{kasatkin}.
Схематическое изображение эксперимента представлено на рисунке \ref{fig:pusher}.
\begin{figure}[h!]
    \centering
    \includegraphics[]{human.png}
    \caption{Cхематическое изображние толкателя и
        положения испытуемого на стабилоплатформе}
    \label{fig:pusher}
\end{figure}

Исходные данные об отклонении сагиттальной коордианты при различных по силе толчках, предоставлены сотрудниками ИМБП РАН (см. рисунок \ref{fig:pushes})
\begin{figure}[h!]
    \centering
    \includegraphics[]{Pushes.png}
    \caption{Отклонение сагиттальной координаты при различных по силе толчках}
    \label{fig:pushes}
\end{figure}

В курсовой работе предполагается рассмотреть возможные
алгоритмы управления изменением позы человека, основанные на решении задачи
оптимального быстродействия, которые можно было бы использовать для
возвращения человека в исходную вертикальную позу. В качестве математической модели
используется модель «перевернутого маятника». В дальнейшем
такое решение предполагается использовать для оценки эффективности управления человеком
при возвращении в вертикальную позу, путем сравнения
времени реального процесса с полученным эталонным решением оптимальной задачи.

\newpage
\section{Математическая модель и постановка задачи управления}
Для описания движения тела человека в сагиттальной плоскости используем традиционную модель перевернутого маятника \cite{falb} (см. рисунок \ref{fig:pendulum}).

\begin{figure}[h!]
    \centering
    \includegraphics[width=1.00\linewidth]{pendulum.png}
    \caption{Модель перевернутого маятника}
    \label{fig:pendulum}
\end{figure}

Традиционно предполагаем, что тело человека в ходе теста допустимо
моделировать недеформируемым однородным стержнем массы $m$,
закрепленным шарнирно в точке $O$, которая соответствует
голеностопному суставу.

Центр масс стержня расположен в точке $C$, удаленной от точки $O$
на расстояние $l$. Момент инерции стержня относительно фронтальной
оси, проходящей через точку $O$, равен $J$. Отклонение стержня от
вертикали описывается углом $\varphi$. Будем считать, что обследуемый
ориентирован так, что его сагиттальная плоскость параллельна оси
чувствительности платформы, а его стопа неподвижна относительно
платформы. Момент $M$, который приложен в точке $O$ к стержню,
будем считать управлением.

Запишем уравнение моментов для малых значений угла $\varphi$ и
скорости его изменения запишем, как традиционно принято для этой задачи.
\begin{equation}\label{6}
    J\ddot{\varphi}= m_Tgl\varphi+M
\end{equation}

Необходимо перевести решение уравнения из начального положения
\begin{equation}\label{7}
    \varphi(0)=\varphi_0, \,\dot{\varphi}(0)=\omega_0
\end{equation}

в конечное положение
\begin{equation}\label{8}
    \varphi(t_k)=\varphi_k,\, \dot{\varphi}(t_k)=0.
\end{equation}

Перевод положения тела должен происходить за минимальное
время $t_k$, с помощью изменений значения момента $M$ в
голеностопном суставе.

Будем принимать во внимание условия ограниченности скорости изменения
момента в голеностопном суставе
\[
    U^-\le\dot{M}\le\ U^+.
\]

Будем считать, что за время толчка нервная система человека
не успела среагировать и момент в голеностопном суставе остался
неизменным и соответствует значению, обеспечивающему положение
равновесия человека до начала движения и после его завершения
\[
    M(0)=M\left(t_k\right)=-m_Tgl\varphi_k;
\]

Для дальнейшего анализа задачи представим приведенные
соотношения в безразмерном виде. Для этого перейдем
к новым переменным
\[
    \theta=\frac{\varphi-\varphi_k}{\varphi_\ast},\ \ m=\frac{M-M_f}{m_Tgl\varphi}.
\]

В качестве характерного значения угла выберем разность
начального и конечного значений угла в голеностопном
суставе при выполнении пробы $\varphi_\ast=\varphi_0-\varphi_k$

Введем безразмерное время
\[
    \tau=\frac{t}{t_\ast},\ t_\ast=\sqrt{\frac{J}{m_Tgl}}.
\]

Управлением $u$ будем считать скорость изменения безразмерного
момента. Для этих переменных обезразмеренные уравнения движения
примут следующий вид
\[
    \theta^{''}=\theta+m;\ m^{'}=u
\]

Здесь через $m^{'}$ обозначено дифференцирование по
безразмерному времени $\tau$. Необходимо решение системы
перевести из начального положения
\begin{equation}\label{9}
    \theta(0)=1;\ \dot{\theta}(0)=\frac{t_\ast}{\varphi_\ast}\omega_0=\Omega_0;\ m(0)=0
\end{equation}
в положение
\begin{equation}\label{9}
    \theta(\tau_f)=0;\ \dot{\theta}(\tau_f)=0;\ m(\tau_f)=0
\end{equation}
с помощью ограниченного управления
\[
    u^-\le\ u\le\ u^+,\ \text{где}
\]
\[
    u^-=\frac{U^-}{mgl\varphi_\ast t_\ast},\ \ u^+=\frac{U^+}{mgl\varphi_\ast t_\ast}.
\]
Далее будем считать, что $u^-=-u^+$
\newpage
\section{Задача оптимального быстродействия при ограничении на величину скорости изменения момента}
Выпишем систему в форме Коши
\begin{equation}\label{koshisystem}
    \left\{ {\begin{aligned}
                 & \theta^{'} = \omega , \hfill   \\
                 & \omega^{'} = \theta+m , \hfill \\
                 & m^{'} = u . \hfill             \\
                 & s^{'} = 1 . \hfill             \\
            \end{aligned}} \right.
\end{equation}
Проверим управляемость системы
\begin{equation*}
    A =
    \begin{pmatrix}
        0 & 1 & 0 \\
        1 & 0 & 1 \\
        0 & 0 & 0
    \end{pmatrix}
    b =
    \begin{pmatrix}
        0 \\
        0 \\
        1
    \end{pmatrix}
    W =
    \begin{pmatrix}
        0 & 0 & 1 \\
        0 & 1 & 0 \\
        1 & 0 & 0
    \end{pmatrix}
\end{equation*}
det $W\neq0$, значит система полностью управляемая

$|u|<U_{max}$
\[
    \theta(0)=1;\ \omega(0)=\frac{t_\ast}{\varphi_\ast}\omega_0=\Omega_0;\ m(0)=0
\]
$J=s(\tau_f)\to min$

Для решения задачи оптимального быстродействия будем использовать принцип максимума Понтрягина \cite{Specpract}:

Если $\{y^0(\cdot),u^0(\cdot),[t_0,t_k]\} - $ оптимальный процесс, то существует нетривиальная пара $\{\lambda_0\geq0,\psi(\cdot)\}$
такая, что
\begin{itemize}
    \item $ \mathop {\max }\limits_{u(t) \in \Omega}  H(\psi(t),y^0(t),u(t))=H(\psi(t),y^0(t),u^0(t))$
          $\forall t \in T \subset [t_0,t_k];$
    \item $\psi(t_k)+\lambda_0(\frac{\partial \varphi_0(y^0(t_k))}{\partial y})^T \perp M \text{ в точке } y^0(t_k);$
    \item $H(t)=H(\psi(t),y^0(t),u^0(t))\equiv0 \text{ при } t \in [t_0,t_k].$
\end{itemize}
Запишем функцию Понтрягина
\[
    H(\Psi(t),y(t),u(t))=\psi_1\cdot\omega+\psi_2\cdot(\theta+m)+\psi_3\cdot u+\psi_4
\]
Сопряженная система уравнений:
\[
    \psi^{'}_i  =  - \frac{{\partial H}}{{\partial y_i }},\,\,i = 1, \ldots ,n
\]
В данной задаче $y_1 = \theta, y_2 = \omega, y_3=m$, тогда сопряженная система примет вид
\begin{equation} \label{7}
    \left\{ {\begin{aligned}
                 & \psi^{'}_1=  - \frac{{\partial H}}{{\partial \theta}} = - \psi_2\hfill  \\
                 & \psi^{'}_2=  - \frac{{\partial H}}{{\partial \omega }} = - \psi_1\hfill \\
                 & \psi^{'}_3=  - \frac{{\partial H}}{{\partial m }} = - \psi_2 \hfill     \\
                 & \psi^{'}_4=- \frac{{\partial H}}{{\partial s }}= 0 \hfill               \\
            \end{aligned}} \right.
\end{equation}
При $\psi_3\equiv0$ следует, что $\psi_2\equiv0$ и $\psi_1\equiv0$, из третьего условия ПМП следует, что и $\psi_4\equiv0$, следовательно особого управления нет.

Тогда для условия максимизации функции Понтрягина
\[
    u=
    \begin{cases}
        -U_{max}, & \text{при $\psi_3<0$}    \\
        +U_{max}, & \text{при $\psi_3\geq0$}
    \end{cases}
\]\\*
Продифференцируем по безразмерному времени второе уравнение из \eqref{7} и подставим в него первое, получим
\[
    \psi{''}_2=\psi_2
\]
Решая систему \eqref{7}, получим
\begin{equation}\label{psisystem}
    \left\{ {\begin{aligned}
                 & \psi_1 = -C_1e^\tau+C_2e^{-\tau}+C_3, \hfill  \\
                 & \psi_2 = C_1e^\tau+C_2e^{-\tau} , \hfill      \\
                 & \psi_3 = -C_1e^\tau+C_2e^{-\tau}+C_3 , \hfill \\
                 & \psi_4 = -C_4 . \hfill                        \\
            \end{aligned}} \right.
\end{equation}

Анализируя корни уравнения $\psi_3(\tau)=0$, для различной комбинации
коэффициентов $C_1,C_2,C_3$, получим, что число корней не может быть больше двух. В системе может быть не более двух переключений $u$.

Пусть $u^*=const$ управление на первом участке траектории до первого переключения $u^*=U_{max}$ или $u^*=U_{min}=-U_{max}$

Решая систему \eqref{koshisystem}, получим
\begin{equation}\label{solvekoshi}
    \left\{ {\begin{aligned}
                 & m(\tau) = u_*\tau+C_0, \hfill                              \\
                 & \theta(\tau) = C_1e^\tau+C_2e^{-\tau}-C_0-u_*\tau , \hfill \\
                 & \omega(\tau) = C_1e^\tau-C_2e^{-\tau}-u_*  . \hfill        \\
            \end{aligned}} \right.
\end{equation}
Пусть первое переключение управления происходит в момент времени
$\tau=\tau_1$, а второе в момент времени
$\tau=\tau_2$. Рассмотрим систему \eqref{koshisystem} на трех этапах,
при переходе между которыми меняется управление.

Этап 1. $u=u_*$ начальные условия
\[
    \theta(0)=1;\ \omega(0)=\Omega_0;\ m(0)=0
\]

Из \eqref{solvekoshi} получим
\begin{equation}\label{stage11}
    \left\{ {\begin{aligned}
                 & 0 = C_0, \hfill                                 \\
                 & 1 = C_1e^\tau+C_2e^{-\tau}-C_0 , \hfill         \\
                 & \Omega_0 = C_1e^\tau-C_2e^{-\tau}-u_*  . \hfill \\
            \end{aligned}} \right.
\end{equation}
Тогда
\begin{equation}\label{stage1Consts}
    \left\{ {\begin{aligned}
                 & C_0 = 0, \hfill                         \\
                 & C_1 = \frac{u_*+1+\Omega_0}{2}, \hfill  \\
                 & C_2 = \frac{u_*-1+\Omega_0}{2} . \hfill \\
            \end{aligned}} \right.
\end{equation}
Подставим полученные константы в \eqref{solvekoshi}
\begin{equation}\label{solvefull1stage}
    \left\{ {\begin{aligned}
                 & m_1(\tau) = u_*\tau, \hfill                                                                              \\
                 & \theta_1(\tau) = \frac{u_*+\Omega_0}{2}(e^\tau-e^{-\tau})+\frac{1}{2}(e^\tau+e^{-\tau})-u_*\tau , \hfill \\
                 & \omega_1(\tau) = \frac{1}{2}(e^\tau-e^{-\tau})+\frac{u_*+\Omega_0}{2}(e^\tau+e^{-\tau})-u_*  . \hfill    \\
            \end{aligned}} \right.
\end{equation}
Этап 2. $u=-u_*$ начальные условия
\[
    \theta(\tau_1)=\theta_1(\tau_1);\ \omega(\tau_1)=\omega_1(\tau_1);\ m(\tau_1)=m_1(\tau_1)
\]
\begin{equation}\label{basecond2stage}
    \left\{ {\begin{aligned}
                 & m(\tau_1) = u_*\tau_1, \hfill                                                                                          \\
                 & \theta(\tau_1) = \frac{u_*+\Omega_0}{2}(e^{\tau_1}-e^{-\tau_1})+\frac{1}{2}(e^{\tau_1}+e^{-\tau_1})-u_*\tau_1 , \hfill \\
                 & \omega(\tau_1) = \frac{1}{2}(e^{\tau_1}-e^{-\tau_1})+\frac{u_*+\Omega_0}{2}(e^{\tau_1}+e^{-\tau_1})-u_*  . \hfill      \\
            \end{aligned}} \right.
\end{equation}
Подставим начальные условия для второго этапа в \eqref{solvekoshi}, получим

\begin{equation}\label{final2stage}
    \left\{ {\begin{aligned}
                 & m_2(\tau) = 2u_*\tau_1-u\tau, \hfill                                                                                                                            \\
                 & \theta_2(\tau) = \frac{u_*+\Omega_0}{2}(e^{\tau_1}-e^{-\tau_1})+\frac{1}{2}(e^{\tau_1}+e^{-\tau_1})-u_*(2\tau_1-e^{\tau_1-\tau}+e^{-\tau_1+\tau}-\tau) , \hfill \\
                 & \omega_2(\tau) = \frac{1}{2}(e^{\tau_1}-e^{-\tau_1})+\frac{u_*+\Omega_0}{2}(e^{\tau_1}+e^{-\tau_1})+u_*(1-e^{\tau_1-\tau}-e^{-\tau_1+\tau})  . \hfill           \\
            \end{aligned}} \right.
\end{equation}
Этап 3. $u=u_*$ конечные условия
\[
    \theta(\tau_f)=0;\ \omega(\tau_f)=0;\ m(\tau_f)=0
\]
Подставим начальные условия в \eqref{solvekoshi}, получим
\begin{equation}\label{stage31}
    \left\{ {\begin{aligned}
                 & 0 = u_*\tau_f+C_0, \hfill                             \\
                 & 0 = C_1e^\tau_f+C_2e^{-\tau_f}-C_0-u_*\tau_f , \hfill \\
                 & 0 = C_1e^\tau_f-C_2e^{-\tau_f}-u_*  . \hfill          \\
            \end{aligned}} \right.
\end{equation}
Тогда решение на этом этапе имеет вид
\begin{equation}\label{final3stage}
    \left\{ {\begin{aligned}
                 & m_3(\tau) = -u_*\tau_f+u_*\tau, \hfill                                                    \\
                 & \theta_3(\tau) = \frac{u_*}{2}(e^{-\tau_f+\tau}-e^{\tau_f-\tau})-u_*(\tau-\tau_f), \hfill \\
                 & \omega_3(\tau) = \frac{u_*}{2}(e^{-\tau_f+\tau}+e^{\tau_f-\tau})-u_*  . \hfill            \\
            \end{aligned}} \right.
\end{equation}
Теперь найдем решения, учитывая что
\begin{equation}\label{T2equalT3}
    \left\{ {\begin{aligned}
                 & m_2(\tau_2) = m_3(\tau_2), \hfill            \\
                 & \theta_2(\tau_2) =  \theta_3(\tau_2), \hfill \\
                 & \omega_2(\tau_2) = \omega_3(\tau_2) . \hfill \\
            \end{aligned}} \right.
\end{equation}
Получим
\[
    \left\{
    \begin{multlined}
        -u_*\tau_f+u_*\tau_2=2u_*\tau_1-u_*\tau_2, \\
        % 2-nd equation
        \shoveleft{ \frac{u_*+\Omega_0}{2}(e^{\tau_1}-e^{-\tau_1})+\frac{1}{2}(e^{\tau_1}+e^{-\tau_1})-u_*(2\tau_1-e^{\tau_1-\tau_2}+e^{-\tau_1+\tau_2}-\tau_2)=} \\
        \shoveright{=\frac{u_*}{2}(e^{-\tau_f+\tau_2}-e^{\tau_f-\tau_2})-u_*(\tau_2-\tau_f),} \\
        % 4-th equation
        \shoveleft{\frac{1}{2}(e^{\tau_1}-e^{-\tau_1})+\frac{u_*+\Omega_0}{2}(e^{\tau_1}+e^{-\tau_1})+u_*(1-e^{\tau_1-\tau_2}-e^{-\tau_1+\tau_2})=} \\
        =\frac{u_*}{2}(e^{-\tau_f+\tau_2}+e^{\tau_f-\tau_2})-u_*.
    \end{multlined}
    \right.
\]

Сократим первое уравнение на $u_*$, выражение для $\tau_f$ из первого уравнения подставим во второе и третье
\[
    \left\{
    \begin{multlined}
        \tau_f=2(\tau_2-\tau_1), \\
        % 2-nd equation
        \shoveleft{\frac{u_*+\Omega_0}{2}(e^{\tau_1}-e^{-\tau_1})+\frac{1}{2}(e^{\tau_1}+e^{-\tau_1})-u_*(-e^{\tau_1-\tau_2}+e^{-\tau_1+\tau_2})=} \\
        \shoveright{=\frac{u_*}{2}(e^{-\tau_f+\tau_2}-e^{\tau_f-\tau_2})+u_*(\tau_f+2(\tau_1-\tau_2))} \\
        % 4-th equation
        \shoveleft{\frac{1}{2}(e^{\tau_1}-e^{-\tau_1})+\frac{u_*+\Omega_0}{2}(e^{\tau_1}+e^{-\tau_1})+u_*(-e^{\tau_1-\tau_2}-e^{-\tau_1+\tau_2})=} \\
        =\frac{u_*}{2}(e^{-\tau_f+\tau_2}+e^{\tau_f-\tau_2})-2u_*.
    \end{multlined}
    \right.
\]
Слагаемое $u_*(\tau_f+2(\tau_1-\tau_2))$ обнуляется
\[
    \left\{
    \begin{multlined}
        \tau_f=2(\tau_2-\tau_1), \\
        % 2-nd equation
        \shoveleft{\frac{u_*+\Omega_0}{2}(e^{\tau_1}-e^{-\tau_1})+\frac{1}{2}(e^{\tau_1}+e^{-\tau_1})-u_*(-e^{\tau_1-\tau_2}+e^{-\tau_1+\tau_2})=} \\
        \shoveright{=\frac{u_*}{2}(e^{-\tau_f+\tau_2}-e^{\tau_f-\tau_2}),} \\
        % 4-th equation
        \shoveleft{\frac{1}{2}(e^{\tau_1}-e^{-\tau_1})+\frac{u_*+\Omega_0}{2}(e^{\tau_1}+e^{-\tau_1})+u_*(-e^{\tau_1-\tau_2}-e^{-\tau_1+\tau_2})=} \\
        =\frac{u_*}{2}(e^{-\tau_f+\tau_2}+e^{\tau_f-\tau_2})-2u_*.
    \end{multlined}
    \right.
\]
Введем замену переменных
\[
    x=e^{\tau_1} ,\,\,y=e^{\tau_2} ,\,\,z=e^{\frac{\tau_f}{2}}
\]
\begin{equation}\label{findtau3}
    \left\{ {\begin{aligned}
                 & z=\frac{y}{x}, \hfill                                                                                                                                         \\
                 & \frac{u_*}{2}\big(\frac{y}{z^2}-\frac{z^2}{y}\big)=\frac{u_*+\Omega_0}{2}(x-\frac{1}{x})+\frac{1}{2}(x+\frac{1}{x})-u_*(\frac{y}{x}-\frac{x}{y}), \hfill      \\
                 & \frac{u_*}{2}\big(\frac{y}{z^2}+\frac{z^2}{y}\big)-2u_*=\frac{1}{2}(x-\frac{1}{x})+\frac{u_*+\Omega_0}{2}(x+\frac{1}{x})-u_*(\frac{y}{x}+\frac{x}{y}). \hfill \\
            \end{aligned}} \right.
\end{equation}
Преобразуем систему к виду
\begin{equation}\label{ytozx}
    \left\{ {\begin{aligned}
                 & y=zx, \hfill                                                                                                                                                      \\
                 & \frac{u_*}{2}\big(\frac{zx}{z^2}-\frac{z^2}{zx}\big)=\frac{u_*+\Omega_0}{2}(x-\frac{1}{x})+\frac{1}{2}(x+\frac{1}{x})-u_*(\frac{zx}{x}-\frac{x}{zx}), \hfill      \\
                 & \frac{u_*}{2}\big(\frac{zx}{z^2}+\frac{z^2}{zx}\big)-2u_*=\frac{1}{2}(x-\frac{1}{x})+\frac{u_*+\Omega_0}{2}(x+\frac{1}{x})-u_*(\frac{zx}{x}+\frac{x}{zx}). \hfill \\
            \end{aligned}} \right.
\end{equation}
\begin{equation}\label{ytozx2}
    \left\{ {\begin{aligned}
                 & y=zx, \hfill                                                                                                                                    \\
                 & \frac{u_*}{2}\big(\frac{x}{z}-\frac{z}{x}\big)=\frac{u_*+\Omega_0}{2}(x-\frac{1}{x})+\frac{1}{2}(x+\frac{1}{x})-u_*(z-\frac{1}{z}), \hfill      \\
                 & \frac{u_*}{2}\big(\frac{x}{z}+\frac{z}{x}\big)-2u_*=\frac{1}{2}(x-\frac{1}{x})+\frac{u_*+\Omega_0}{2}(x+\frac{1}{x})-u_*(z+\frac{1}{z}). \hfill \\
            \end{aligned}} \right.
\end{equation}
Сложим и вычтем 2 и 3 уравнения из \eqref{ytozx2}, получим
\begin{equation}\label{ytozx3}
    \left\{ {\begin{aligned}
                 & y=zx, \hfill                                                                                                               \\
                 & u_*\frac{z}{x}+2u_*=\frac{1-u_*-\Omega_0}{2}(x-\frac{1}{x})+\frac{u_*+\Omega_0-1}{2}(x+\frac{1}{x})-2\frac{u_*}{z}, \hfill \\
                 & u_*\frac{x}{z}-2u_*=\frac{u_*+\Omega_0+1}{2}(x-\frac{1}{x})+\frac{u_*+\Omega_0+1}{2}(x+\frac{1}{x})-2u_*z. \hfill          \\
            \end{aligned}} \right.
\end{equation}
Выразим из второго уравнения переменную $x$
\[
    x=\frac{z(-1+u-uz+\Omega_0)}{2u(1+z)}
\]
И подставим в третье, получим
\begin{equation}\label{fullquadratic1}
    \footnotesize
    \left\{ {\begin{aligned}
                 & z=\frac{-1+2u_*^2+2u_*\Omega_0+\Omega_0^2-\sqrt{32u_*^3+64u_*^4+4u_*^2\Omega_0^2+4u_*\Omega_0(\Omega_0^2-1)+(\Omega_0^2-1)^2}}{2u_*(1+5u_*+\Omega_0)}, \hfill                          \\
                 & x=\frac{-5-8u_*(2+3u_*)+10u_*\Omega_0+5\Omega_0^2-3\sqrt{1+32u_*^3(1+2u_*)-4u_*\Omega_0+2(2u_*^2-1)\Omega_0^2+4u_*\Omega_0^3+\Omega_0^4}}{4(1+2u_*+\Omega_0)(1+5u_*+\Omega_0)}. \hfill \\
            \end{aligned}} \right .
\end{equation}
или
\begin{equation}\label{fullquadratic2}
    \footnotesize
    \left\{ {\begin{aligned}
                 & z=\frac{-1+2u_*^2+2u_*\Omega_0+\Omega_0^2+\sqrt{32u_*^3+64u_*^4+4u_*^2\Omega_0^2+4u_*\Omega_0(\Omega_0^2-1)+(\Omega_0^2-1)^2}}{2u_*(1+5u_*+\Omega_0)}, \hfill                          \\
                 & x=\frac{-5-8u_*(2+3u_*)+10u_*\Omega_0+5\Omega_0^2+3\sqrt{1+32u_*^3(1+2u_*)-4u_*\Omega_0+2(2u_*^2-1)\Omega_0^2+4u_*\Omega_0^3+\Omega_0^4}}{4(1+2u_*+\Omega_0)(1+5u_*+\Omega_0)}. \hfill \\
            \end{aligned}} \right .
\end{equation}
Так как мы ищем $\tau_f$ наименьшее, то нужно отобрать наименьшее положительное $z>1$.
Таким условиям удовлетворяет система \eqref{fullquadratic1}.
% Тогда из \eqref{findtau} получим
% \begin{equation}\label{xyz}
%     \footnotesize\left\{ {\begin{aligned}
%                  & , \hfill              \\
%                  & \hfill                \\
%                  & 2z=\frac{y^2}{x^2}. \hfill \\
%             \end{aligned}} \right.
% \end{equation}
\section{Сравнение с экспериментальными данными}
Построим оптимальную траекторию, учитывая полученные соотношения для управления.

$m_T=74$кг; $l=0.88$м; $J=\frac{4ml^2}{3}$кг$\cdot$м$^2$
\[
    t_\ast=\sqrt{\frac{J}{m_Tgl}}=\sqrt{\frac{4l}{3g}}=0.346c
\]
$U_{max}=35\text{н$\cdot$м/с$^2$}$; $\omega_0=\frac{2\pi}{9}$; $\varphi_\ast=\frac{\pi}{9}; \Omega_0=\frac{t_*}{\varphi_*}\omega_0=0.692$
\[
    u^*=1.65
\]
$z=0.979371$
\newpage
\section{Заключение}
В данной работе была рассмотрена задача оптимального
\newpage
\addcontentsline{toc}{section}{Список используемой литературы}
\begin{thebibliography}{15}
    \bibitem{Specpract} Александров В.В. Спецпрактикум по теоретической и прикладной
    механике, Издательство Московского университета, 2009, 234 с.
    \bibitem{Optimal} Понтрягин Л.С., Болтянский В.Г., Гамкрелидзе Р.В., Мищенко Е.Ф. Математическая теория оптимальных процессов. Москва, Наука, 1983, 393 с.
    \bibitem{PAKrychinin}Кручинин П.А. Анализ результатов стабилометрических тестов со ступенчатым воздействием с точки зрения механики управляемых систем
    // Биофизика. – 2019. – Т. 64, №5. – С. 1–11.
    \bibitem{kasatkin} П. А. Кручинин и Е. А. Касаткин, Изв. ЮФУ.
    Техн. науки 10 (159), 254 (2014).
    \bibitem{falb}М. Атанс и П. Фалб, Оптимальное управление
    (Машиностроение, М., 1968).
\end{thebibliography}
\end{document}